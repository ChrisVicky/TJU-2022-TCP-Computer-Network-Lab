% !TEX root = ../tjumain.tex
\chapter{总结}

在本次实践中,我们顺利完成了每次任务,并拿到了每次 AutoLab 的满分。但在其中我们也遇到了不少困难。

\paragraph*{环境配置} 由于本人使用习惯需要在 Arch Linux 环境下进行实验,于是在了解 vagrant virtualbox 等环境作用原理之后,最终成功在 arch linux 环境下完成了实验。

\paragraph*{tree} 在设计接收乱序缓冲区时,我们首先考虑了直接使用 recv\_buf 和一些指针进行操作。但这样的操作错误率太高,且耦合性过大,需要在每次放入、取出时频繁操作指针,不利于后续的维护更新。于是我们不仅将放入、取出的操作进行了封装。同时,由于考虑到这个实际应用的操作逻辑,结合自身在 ACM 中学到的数据结构,我果断考虑到了小根堆,但由于是 C 的环境,于是不得不手写一个小根堆,并做一层封装。

\paragraph*{反复重复测试} 在拥塞控制的测试中,我们始终难以得到心仪的结果,于是我们利用家里的一台闲置的主机,临时配置了 Arch Linux 和一些基础设施,通过 ssh -R 连接到服务器上,于是我们能够在学校也远程使用。最终我们编写脚本,在机器上跑了3天完成了1700多个测试结果,再使用vim及其脚本编写了一个html用于,在浏览器上找到了比较出色的几个结果用于报告。

总的来说,这次实践加深了我对 TCP 协议的理解和一些思考,也让我们能更加理性地对日常遇到的网络问题想出正确的策略。但仍有很多网络现象是我们尚不能解释的:例如不能在校园网环境下通过 ssh 连接另一台电脑等等。对于这些现象,我们十分感兴趣,在未来的日子里我们也会继续提升对计算机网络的认识和理解,争取回答这些问题。
