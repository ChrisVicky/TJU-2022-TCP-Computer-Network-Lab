% !TEX root = ../tjumain.tex
\chapter{可靠数据传输的实现}

\section{Client 端}
\subsection{Tju\_Send 实现}

在 Tju\_Send 中,首先需要判断上层内容的大小,若过大,则需要进行切片,以免在下层被切割。然后在进入阻塞状态,等待发送窗口出现空闲。然后交给发送窗口,同时调用发送函数,在 rwnd 的情况下将发送缓冲区的内容进行发送(此处在后期需要调整为 线程)

\section{Server 端}
\subsection{Tju\_Recv 实现}
在 Tju\_Recv 中,首先进行阻塞等待,直到其他线程给接收缓冲区放入内容。待有内容后,再将内容 通过 Memcpy 提交给上层调用的实体。

\section{通用}
\subsection{Socket 数据结构}
\begin{itemize}
    \item 给 window.wnd\_send 增加了 rwnd 和 cwnd 等控制发送的信息
    \item 给 window.wnd\_recv 增加了 AVL Tree 等数据结构,用于存放乱序到达的数据包
    \item 给 window.wnd\_send 增加了 rto 等字段,动态控制超时事件
    \item 给 window.wnd\_recv 增加了 expe\_seq 等字段,判断接收到的数据顺序正确与否
\end{itemize}

\subsection{超时重传机制}
\subsubsection*{超时重传定时器}
实现了 Timer\_Helpher 子系统,通过设置 set\_timer() 函数来新建一个 Timer,Delete\_timer() 来终止一个 Timer。使用了一个线程不断检查是否有超时的Timer,并进行重传和重置。使用 Mutex Lock 进行存储区域的一致性,使得在增加、删除和检查时只能有一个线程存在。

特别的,我们使用了链表的数据结构进行 Timer 的增加、删除和检查,链表能够高效地进行增加删除。同时,设置了 event 数据类型来处理当前 Timer 的超时操作(可以通过传入不同的函数和变量达到不同类型 Timer 的统一调用)。

\subsubsection*{RTO调整}

由于我们每个需要重传的报文都对应一个 TIMER,所以我们能通过发送时的 Created\_At 和删除时的系统事件算出当前 Timer 提供的 sampleRTT 并 在每次进行 Timer 删除时,通过公式 \ref{eq:rto} 进行 RTO 的更新。

值得注意的是,这里依然需要大量使用 Mutex Locker 来控制:在更新 RTO 时,不能创建新的Timer,直到 RTO 更新完成。

\subsubsection*{流量控制}

接收方发送ACK时,将自己缓冲区大小放入 advertised\_wind 字段。发送方在接收到 ACK 时,需要提取 rwnd 值。在设置发送窗口的大小时,我们需要考虑 rwnd(发送窗口大小), cwnd(拥塞控制), 以及自己本身的大小。

当得到 rwnd == 0 的情况时,发送方需要发送1比特的试探报文进行发送窗口的试探,同时需要设置超时机制进行不断试探。

\subsubsection*{日志记录}

实现了日志 trace 模块,在Server Client 端调用 tju\_sock 的时候进行日志的初始化(即定义日志名,创建日志文件等)。然后通过提供的日志格式和事件,在相应事件发生时,调用响应函数进行日志的录入。



